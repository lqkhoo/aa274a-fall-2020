\documentclass{article}

\usepackage{amsmath}
\usepackage{amsthm}
\usepackage{amssymb}
\usepackage{bm}
\usepackage{bbm}
\usepackage{fancyhdr}
% \usepackage{listings}
\usepackage{cite}
\usepackage{graphicx}
\usepackage{enumitem}
\usepackage{courier}
\usepackage[pdftex,colorlinks=true, urlcolor = blue]{hyperref}
\usepackage{pdfpages}


\oddsidemargin 0in \evensidemargin 0in
\topmargin -0.5in \headheight 0.25in \headsep 0.25in
\textwidth 6.5in \textheight 9in
\parskip 6pt \parindent 0in \footskip 20pt

% set the header up
\fancyhead{}
\fancyhead[L]{Stanford Aeronautics \& Astronautics}
\fancyhead[R]{Fall 2020}

%%%%%%%%%%%%%%%%%%%%%%%%%%
\renewcommand\headrulewidth{0.4pt}
\setlength\headheight{15pt}

\usepackage{xparse}
\NewDocumentCommand{\codeword}{v}{%
\texttt{\textcolor{blue}{#1}}%
}

\usepackage{xcolor}
\setlength{\parindent}{0in}

\title{AA 274A: Principles of Robot Autonomy I \\ Problem Set 2}
\author{Name: Li Quan Khoo     \\ SUID: lqkhoo (06154100)}
\date{\today}

\begin{document}

\maketitle
\pagestyle{fancy} 

\section*{Problem 1: Camera Calibration}
\begin{enumerate}[label=(\roman*)]
\item % (i)
(code)

\item % (ii)
(code)

\item % (iii)
(code)

\item % (iv)
(code)

\item % (v)
(code)

\end{enumerate}


\section*{Problem 2: Line Extraction}
\begin{enumerate}[label=(\roman*)]
\item % (i)
(code)

\item % (ii)
TODO

\end{enumerate}


\section*{Problem 3: Linear Filtering}
\begin{enumerate}[label=(\roman*)]
\item % (i)
TODO

\item % (ii)
TODO

\item % (iii)
(code)

\item % (iv)
TODO

\item % (v)
TODO

\item % (vi)
(code)

\item % (vii)
TODO
	
\end{enumerate}


\section*{Problem 4: Template Matching}
\begin{enumerate}[label=(\roman*)]
\item % (i)
(code)

\item % (ii)
(code)

\item % (iii)
TODO
	
\end{enumerate}


\section*{Problem 5: Stop Sign Detection and FSM in ROS}
\begin{enumerate}[label=(\roman*)]
\item % (i)
TODO

\item % (ii)
(code)

\item % (iii)
TODO

\item % (iv)
TODO

\item % (v)
null

\item % (vi)
TODO

\item % (vii)
(code)

\item % (viii)
TODO
	
\end{enumerate}

\section*{Extra Problem: Image Pyramids}
\begin{enumerate}[label=(\roman*)]
\item % (i)
(code)

\item % (ii)
TODO

\item % (iii)
(code)

\item % (iv)
TODO

\item % (v)
(code)

\item % (vi)
TODO

\item % (vii)
(code)

\item % (viii)
(code)

\item % (xi)
TODO

\end{enumerate}

\end{document}
